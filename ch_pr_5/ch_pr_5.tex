\documentclass[journal,12pt,twocolumn]{IEEEtran}

\usepackage{setspace}
\usepackage{gensymb}
\singlespacing
\usepackage[cmex10]{amsmath}

\usepackage{amsthm}

\usepackage{mathrsfs}
\usepackage{txfonts}
\usepackage{stfloats}
\usepackage{bm}
\usepackage{cite}
\usepackage{cases}
\usepackage{subfig}

\usepackage{longtable}
\usepackage{multirow}
\usepackage{enumitem}
\usepackage{mathtools}
\usepackage{steinmetz}
\usepackage{tikz}
\usepackage{circuitikz}
\usepackage{verbatim}
\usepackage{tfrupee}
\usepackage[breaklinks=true]{hyperref}
\usepackage{graphicx}
\usepackage{tkz-euclide}

\usetikzlibrary{calc,math}
\usepackage{listings}
    \usepackage{color}                                            %%
    \usepackage{array}                                            %%
    \usepackage{longtable}                                        %%
    \usepackage{calc}                                             %%
    \usepackage{multirow}                                         %%
    \usepackage{hhline}                                           %%
    \usepackage{ifthen}                                           %%
    \usepackage{lscape}     
\usepackage{multicol}
\usepackage{chngcntr}

\DeclareMathOperator*{\Res}{Res}

\renewcommand\thesection{\arabic{section}}
\renewcommand\thesubsection{\thesection.\arabic{subsection}}
\renewcommand\thesubsubsection{\thesubsection.\arabic{subsubsection}}

\renewcommand\thesectiondis{\arabic{section}}
\renewcommand\thesubsectiondis{\thesectiondis.\arabic{subsection}}
\renewcommand\thesubsubsectiondis{\thesubsectiondis.\arabic{sub subsection}}


\hyphenation{optical networks semiconduc-tor}
\def\inputGnumericTable{}                                 %%

\lstset{
%language=C,
frame=single, 
breaklines=true,
columns=fullflexible
}
\date{March 2021}

\begin{document}

\newcommand{\BEQA}{\begin{eqnarray}}
\newcommand{\EEQA}{\end{eqnarray}}
\newcommand{\define}{\stackrel{\triangle}{=}}
\bibliographystyle{IEEEtran}
\raggedbottom
\setlength{\parindent}{0pt}
\providecommand{\mbf}{\mathbf}
\providecommand{\pr}[1]{\ensuremath{\Pr\left(#1\right)}}
\providecommand{\qfunc}[1]{\ensuremath{Q\left(#1\right)}}
\providecommand{\fn}[1]{\ensuremath{f\left({#1}\right)}}
\providecommand{\e}[1]{\ensuremath{E\left(#1\right)}}
\providecommand{\sbrak}[1]{\ensuremath{{}\left[#1\right]}}
\providecommand{\lsbrak}[1]{\ensuremath{{}\left[#1\right.}}
\providecommand{\rsbrak}[1]{\ensuremath{{}\left.#1\right]}}
\providecommand{\brak}[1]{\ensuremath{\left(#1\right)}}
\providecommand{\lbrak}[1]{\ensuremath{\left(#1\right.}}
\providecommand{\rbrak}[1]{\ensuremath{\left.#1\right)}}
\providecommand{\cbrak}[1]{\ensuremath{\left\{#1\right\}}}
\providecommand{\lcbrak}[1]{\ensuremath{\left\{#1\right.}}
\providecommand{\rcbrak}[1]{\ensuremath{\left.#1\right\}}}
\theoremstyle{remark}
\newtheorem{rem}{Remark}
\newcommand{\sgn}{\mathop{\mathrm{sgn}}}
\newcommand{\comb}[2]{{}^{#1}\mathrm{C}_{#2}}
\providecommand{\abs}[1]{\vert#1\vert}
\providecommand{\res}[1]{\Res\displaylimits_{#1}} 
\providecommand{\norm}[1]{\lVert#1\rVert}
%\providecommand{\norm}[1]{\lVert#1\rVert}
\providecommand{\mtx}[1]{\mathbf{#1}}
\providecommand{\mean}[1]{E\sbrak{ #1 }}
\providecommand{\fourier}{\overset{\mathcal{F}}{ \rightleftharpoons}}
%\providecommand{\hilbert}{\overset{\mathcal{H}}{ \rightleftharpoons}}
\providecommand{\system}{\overset{\mathcal{H}}{ \longleftrightarrow}}
	%\newcommand{\solution}[2]{\textbf{Solution:}{#1}}
\newcommand{\solution}{\noindent \textbf{Solution: }}
\newcommand{\cosec}{\,\text{cosec}\,}
\providecommand{\dec}[2]{\ensuremath{\overset{#1}{\underset{#2}{\gtrless}}}}
\newcommand{\myvec}[1]{\ensuremath{\begin{pmatrix}#1\end{pmatrix}}}
\newcommand{\mydet}[1]{\ensuremath{\begin{vmatrix}#1\end{vmatrix}}}
\numberwithin{equation}{subsection}
\makeatletter
\vspace{3cm}
\title{Challenging question 5}
\author{Adhvik Mani Sai Murarisetty - AI20BTECH11015}
\maketitle
\newpage
\bigskip
\renewcommand{\thetable}{\theenumi}

%
Download latex-tikz codes from 
%
\begin{lstlisting}
https://github.com/adhvik24/AI1103-PROBABILITY-AND-RANDOM-VARIABLES/blob/main/ch_pr_5/ch_pr_5.tex
\end{lstlisting}
\section{Challenging question 5}
Suppose $X_1,X_2,X_3$ and $X_4$ are independent and identically distributed random variables, having density function f. Then,
\begin{enumerate}
\item \pr{X_4 > Max(X_1,X_2) > X_3} = $\frac{1}{6}$
\item \pr{X_4 > Max(X_1,X_2) > X_3} = $\frac{1}{8}$
\item \pr{X_4 > X_3 > Max(X_1,X_2)} = $\frac{1}{12}$
\item \pr{X_4 > X_3 > Max(X_1,X_2)} = $\frac{1}{6}$
\end{enumerate}
\section{SOLUTION}
Total number of possible arrangements of these four R.V's in an order = $4!$=24 ways.

As given $X_1,X_2,X_3$ and $X_4$ are independent and identically distributed random variables they all are equally likely for a position.

Required orders are:
\begin{enumerate}
    \item $X_4 > Max(X_1,X_2) > X_3$, for this possible orders are: 
\begin{enumerate}
    \item $X_4 > X_1 > X_2 > X_3$
    \item $X_4 > X_1 > X_3 > X_2$
    \item $X_4 > X_2 > X_1 > X_3$
    \item $X_4 > X_2 > X_3 > X_1$
\end{enumerate}
    Therefore four possibilities. So,
    \begin{align}
        \pr{X_4 > Max(X_1,X_2) > X_3}&=\frac{4}{24}\\
        &=\frac{1}{6}
    \end{align}
    \item $X_4 > X_3 > Max(X_1,X_2)$, for this possible orders are: 
\begin{enumerate}
    \item $X_4 > X_3 > X_1 > X_2$
    \item $X_4 > X_3 > X_2 > X_1$
\end{enumerate}
Therefore two possibilities. So,
\begin{align}
    \pr{X_4 > X_3 > Max(X_1,X_2)}&=\frac{2}{24}\\
    &=\frac{1}{12}
\end{align}
\end{enumerate}
Therefore correct answers are \textbf{(1) and (3)}.
\end{document}
