\documentclass[journal,12pt,twocolumn]{IEEEtran}

\usepackage{setspace}
\usepackage{gensymb}
\singlespacing
\usepackage[cmex10]{amsmath}

\usepackage{amsthm}

\usepackage{mathrsfs}
\usepackage{txfonts}
\usepackage{stfloats}
\usepackage{bm}
\usepackage{cite}
\usepackage{cases}
\usepackage{subfig}

\usepackage{longtable}
\usepackage{multirow}
\usepackage{enumitem}
\usepackage{mathtools}
\usepackage{steinmetz}
\usepackage{tikz}
\usepackage{circuitikz}
\usepackage{verbatim}
\usepackage{tfrupee}
\usepackage[breaklinks=true]{hyperref}
\usepackage{graphicx}
\usepackage{tkz-euclide}

\usetikzlibrary{calc,math}
\usepackage{listings}
    \usepackage{color}                                            %%
    \usepackage{array}                                            %%
    \usepackage{longtable}                                        %%
    \usepackage{calc}                                             %%
    \usepackage{multirow}                                         %%
    \usepackage{hhline}                                           %%
    \usepackage{ifthen}                                           %%
    \usepackage{lscape}     
\usepackage{multicol}
\usepackage{chngcntr}

\DeclareMathOperator*{\Res}{Res}

\renewcommand\thesection{\arabic{section}}
\renewcommand\thesubsection{\thesection.\arabic{subsection}}
\renewcommand\thesubsubsection{\thesubsection.\arabic{subsubsection}}

\renewcommand\thesectiondis{\arabic{section}}
\renewcommand\thesubsectiondis{\thesectiondis.\arabic{subsection}}
\renewcommand\thesubsubsectiondis{\thesubsectiondis.\arabic{sub subsection}}


\hyphenation{optical networks semiconduc-tor}
\def\inputGnumericTable{}                                 %%

\lstset{
%language=C,
frame=single, 
breaklines=true,
columns=fullflexible
}
\date{March 2021}

\begin{document}

\newcommand{\BEQA}{\begin{eqnarray}}
\newcommand{\EEQA}{\end{eqnarray}}
\newcommand{\define}{\stackrel{\triangle}{=}}
\bibliographystyle{IEEEtran}
\raggedbottom
\setlength{\parindent}{0pt}
\providecommand{\mbf}{\mathbf}
\providecommand{\pr}[1]{\ensuremath{\Pr\left(#1\right)}}
\providecommand{\qfunc}[1]{\ensuremath{Q\left(#1\right)}}
\providecommand{\fn}[1]{\ensuremath{f\left({#1}\right)}}
\providecommand{\e}[1]{\ensuremath{E\left(#1\right)}}
\providecommand{\sbrak}[1]{\ensuremath{{}\left[#1\right]}}
\providecommand{\lsbrak}[1]{\ensuremath{{}\left[#1\right.}}
\providecommand{\rsbrak}[1]{\ensuremath{{}\left.#1\right]}}
\providecommand{\brak}[1]{\ensuremath{\left(#1\right)}}
\providecommand{\lbrak}[1]{\ensuremath{\left(#1\right.}}
\providecommand{\rbrak}[1]{\ensuremath{\left.#1\right)}}
\providecommand{\cbrak}[1]{\ensuremath{\left\{#1\right\}}}
\providecommand{\lcbrak}[1]{\ensuremath{\left\{#1\right.}}
\providecommand{\rcbrak}[1]{\ensuremath{\left.#1\right\}}}
\theoremstyle{remark}
\newtheorem{rem}{Remark}
\newcommand{\sgn}{\mathop{\mathrm{sgn}}}
\newcommand{\comb}[2]{{}^{#1}\mathrm{C}_{#2}}
\providecommand{\abs}[1]{\vert#1\vert}
\providecommand{\res}[1]{\Res\displaylimits_{#1}} 
\providecommand{\norm}[1]{\lVert#1\rVert}
%\providecommand{\norm}[1]{\lVert#1\rVert}
\providecommand{\mtx}[1]{\mathbf{#1}}
\providecommand{\mean}[1]{E\sbrak{ #1 }}
\providecommand{\fourier}{\overset{\mathcal{F}}{ \rightleftharpoons}}
%\providecommand{\hilbert}{\overset{\mathcal{H}}{ \rightleftharpoons}}
\providecommand{\system}{\overset{\mathcal{H}}{ \longleftrightarrow}}
	%\newcommand{\solution}[2]{\textbf{Solution:}{#1}}
\newcommand{\solution}{\noindent \textbf{Solution: }}
\newcommand{\cosec}{\,\text{cosec}\,}
\providecommand{\dec}[2]{\ensuremath{\overset{#1}{\underset{#2}{\gtrless}}}}
\newcommand{\myvec}[1]{\ensuremath{\begin{pmatrix}#1\end{pmatrix}}}
\newcommand{\mydet}[1]{\ensuremath{\begin{vmatrix}#1\end{vmatrix}}}
\numberwithin{equation}{subsection}
\makeatletter
\vspace{3cm}
\title{Assignment 8}
\author{Adhvik Mani Sai Murarisetty - AI20BTECH11015}
\maketitle
\newpage
\bigskip
\renewcommand{\thetable}{\theenumi}

%
Download latex-tikz codes from 
%
\begin{lstlisting}
https://github.com/adhvik24/AI1103-PROBABILITY-AND-RANDOM-VARIABLES/tree/main/ASSIGNMENT_8/AI1103_Assignment8.tex
\end{lstlisting}
\section{ CSIR UGC NET EXAM (June  2013), Q.71}
Let X be a random variable with probability density function,
\begin{align}
    f(x)=\alpha(x-\mu)^{\alpha-1}e^{-(x-\mu)^{\alpha}}
\end{align}
such that $-\infty<\mu<\infty\;;\alpha>0\;;x>\mu$, The hazard function is: 
\begin{enumerate}
    \item constant for all $\alpha$
    \item an increasing function for some $\alpha$
    \item independent of $\alpha$
    \item independent of $\mu$ when $\alpha=1$
\end{enumerate}
\section{SOLUTION}
Given PDF of X,
\begin{align}
    f(x)=\alpha(x-\mu)^{\alpha-1}e^{-(x-\mu)^{\alpha}}\label{3}
\end{align}
\textbf{Important property}(using in \eqref{1} as $x>\mu$):

Given $x-y>0$ and $-\infty<y<\infty$, then
\begin{align}
    \lim_{x \to -\infty} x-y=0
\end{align}
CDF of X,
\begin{align}
    F(x)&=\int_{-\infty}^x{f(x)\;dx}\\
    &=\int_{-\infty}^x{\alpha(x-\mu)^{\alpha-1}e^{-(x-\mu)^{\alpha}}dx}\\
    &=\int_{-\infty}^x{e^{-(x-\mu)^{\alpha}}d(x-\mu)^{\alpha}}\\
    &=\sbrak{\frac{e^{-(x-\mu)^{\alpha}}}{-1}}_{-\infty}^x\\
    &=-e^{-(x-\mu)^{\alpha}}-\lim_{x \to -\infty} \frac{e^{-(x-\mu)^{\alpha}}}{-1}\label{1}\\
    &=-e^{-(x-\mu)^{\alpha}}+ {e^{-(0)^{\alpha}}}\\
   F(x) &=1-e^{-(x-\mu)^{\alpha}}\label{2}
\end{align}
Hazard function $\beta(x)$,(using \eqref{3} and \eqref{2})
\begin{align}
    \beta(x)&=\frac{f(x)}{1-F(x)}\\
    &=\frac{\alpha(x-\mu)^{\alpha-1}e^{-(x-\mu)^{\alpha}}}{1-(1-e^{-(x-\mu)^{\alpha}})}\\
    &=\frac{\alpha(x-\mu)^{\alpha-1}e^{-(x-\mu)^{\alpha}}}{e^{-(x-\mu)^{\alpha}}}\\
    \beta(x)&=\alpha(x-\mu)^{\alpha-1}
\end{align}
\begin{enumerate}
    \item $\beta(x)$ is not constant for all $\alpha$
    \item $\beta(x)=\alpha(x-\mu)^{\alpha-1}$ is an increasing function for $\alpha<0 \;or\; \alpha>1$ as given $x-\mu>0$ for all x.
    
    \textbf{Proof: }
    Using first derivative test, A function is increasing iff its first derivative is positive for all x. 
    \begin{align}
        \dfrac{d}{dx} \beta(x)&=  \dfrac{d}{dx} \alpha(x-\mu)^{\alpha-1}\\
        &= \alpha(\alpha-1)(x-\mu)^{\alpha-2}\label{0}
    \end{align}
    For \eqref{0} to be positive, (As given $x-\mu>0$)
    \begin{align}
        \alpha(\alpha-1)(x-\mu)^{\alpha-2}&>0\\
        \alpha(\alpha-1)&>0\\
        \implies \alpha \in \brak{-\infty,0}\cup \brak{1,\infty}
    \end{align}
    $\therefore \beta(x)$ an increasing function for some $\alpha$
    \item $\beta(x)$ is dependent of $\alpha$
    \item when $\alpha=1$,
\begin{align}
    \beta(x)&=\alpha(x-\mu)^{0}=\alpha
\end{align}
Therefore the hazard function is independent of $\mu$ when $\alpha=1$.
\end{enumerate}
\textbf{ANSWER: (2) and (4)}
\end{document}
