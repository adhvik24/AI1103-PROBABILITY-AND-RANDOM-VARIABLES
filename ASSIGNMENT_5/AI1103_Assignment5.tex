\documentclass[journal,12pt,twocolumn]{IEEEtran}

\usepackage{setspace}
\usepackage{gensymb}
\singlespacing
\usepackage[cmex10]{amsmath}

\usepackage{amsthm}

\usepackage{mathrsfs}
\usepackage{txfonts}
\usepackage{stfloats}
\usepackage{bm}
\usepackage{cite}
\usepackage{cases}
\usepackage{subfig}

\usepackage{longtable}
\usepackage{multirow}
\usepackage{enumitem}
\usepackage{mathtools}
\usepackage{steinmetz}
\usepackage{tikz}
\usepackage{circuitikz}
\usepackage{verbatim}
\usepackage{tfrupee}
\usepackage[breaklinks=true]{hyperref}
\usepackage{graphicx}
\usepackage{tkz-euclide}

\usetikzlibrary{calc,math}
\usepackage{listings}
    \usepackage{color}                                            %%
    \usepackage{array}                                            %%
    \usepackage{longtable}                                        %%
    \usepackage{calc}                                             %%
    \usepackage{multirow}                                         %%
    \usepackage{hhline}                                           %%
    \usepackage{ifthen}                                           %%
    \usepackage{lscape}     
\usepackage{multicol}
\usepackage{chngcntr}

\DeclareMathOperator*{\Res}{Res}

\renewcommand\thesection{\arabic{section}}
\renewcommand\thesubsection{\thesection.\arabic{subsection}}
\renewcommand\thesubsubsection{\thesubsection.\arabic{subsubsection}}

\renewcommand\thesectiondis{\arabic{section}}
\renewcommand\thesubsectiondis{\thesectiondis.\arabic{subsection}}
\renewcommand\thesubsubsectiondis{\thesubsectiondis.\arabic{sub subsection}}


\hyphenation{optical networks semiconduc-tor}
\def\inputGnumericTable{}                                 %%

\lstset{
%language=C,
frame=single, 
breaklines=true,
columns=fullflexible
}
\date{March 2021}

\begin{document}

\newcommand{\BEQA}{\begin{eqnarray}}
\newcommand{\EEQA}{\end{eqnarray}}
\newcommand{\define}{\stackrel{\triangle}{=}}
\bibliographystyle{IEEEtran}
\raggedbottom
\setlength{\parindent}{0pt}
\providecommand{\mbf}{\mathbf}
\providecommand{\pr}[1]{\ensuremath{\Pr\left(#1\right)}}
\providecommand{\qfunc}[1]{\ensuremath{Q\left(#1\right)}}
\providecommand{\fn}[1]{\ensuremath{f\left({#1}\right)}}
\providecommand{\e}[1]{\ensuremath{E\left(#1\right)}}
\providecommand{\sbrak}[1]{\ensuremath{{}\left[#1\right]}}
\providecommand{\lsbrak}[1]{\ensuremath{{}\left[#1\right.}}
\providecommand{\rsbrak}[1]{\ensuremath{{}\left.#1\right]}}
\providecommand{\brak}[1]{\ensuremath{\left(#1\right)}}
\providecommand{\lbrak}[1]{\ensuremath{\left(#1\right.}}
\providecommand{\rbrak}[1]{\ensuremath{\left.#1\right)}}
\providecommand{\cbrak}[1]{\ensuremath{\left\{#1\right\}}}
\providecommand{\lcbrak}[1]{\ensuremath{\left\{#1\right.}}
\providecommand{\rcbrak}[1]{\ensuremath{\left.#1\right\}}}
\theoremstyle{remark}
\newtheorem{rem}{Remark}
\newcommand{\sgn}{\mathop{\mathrm{sgn}}}
\newcommand{\comb}[2]{{}^{#1}\mathrm{C}_{#2}}
\newcommand{\p}[2]{P\left(X={#1} , Y={#2}\right)}
\newcommand{\q}[1]{P\left(X={#1}\right)}
\providecommand{\abs}[1]{\vert#1\vert}
\providecommand{\res}[1]{\Res\displaylimits_{#1}} 
\providecommand{\norm}[1]{\lVert#1\rVert}
%\providecommand{\norm}[1]{\lVert#1\rVert}
\providecommand{\mtx}[1]{\mathbf{#1}}
\providecommand{\mean}[1]{E[ #1 ]}
\providecommand{\fourier}{\overset{\mathcal{F}}{ \rightleftharpoons}}
%\providecommand{\hilbert}{\overset{\mathcal{H}}{ \rightleftharpoons}}
\providecommand{\system}{\overset{\mathcal{H}}{ \longleftrightarrow}}
	%\newcommand{\solution}[2]{\textbf{Solution:}{#1}}
\newcommand{\solution}{\noindent \textbf{Solution: }}
\newcommand{\cosec}{\,\text{cosec}\,}
\providecommand{\dec}[2]{\ensuremath{\overset{#1}{\underset{#2}{\gtrless}}}}
\newcommand{\myvec}[1]{\ensuremath{\begin{pmatrix}#1\end{pmatrix}}}
\newcommand{\mydet}[1]{\ensuremath{\begin{vmatrix}#1\end{vmatrix}}}
\numberwithin{equation}{subsection}
\makeatletter
\vspace{3cm}
\title{Assignment 5}
\author{Adhvik Mani Sai Murarisetty - AI20BTECH11015}
\maketitle
\newpage
\bigskip
\renewcommand{\thetable}{\theenumi}

%
Download latex-tikz codes from 
%
\begin{lstlisting}
https://github.com/adhvik24/AI1103-PROBABILITY-AND-RANDOM-VARIABLES/tree/main/ASSIGNMENT_5/AI1103_Assignment5.tex
\end{lstlisting}
\section{GATE 2019 (ST) , Q.45 (statistics section)}
Consider the trinomial distribution with the probability mass function 
\begin{multline}
    \nonumber \p{x}{y}\\=\brak{\frac{7!}{x!y!(7-x-y)!}}(0.6)^x(0.2)^y(0.2)^{7-x-y}
\end{multline}
where $x\geq0 , y\geq0 \;and\; {x+y}\leq7$.
Then $\mean{Y|X=3}$ is equal to
\section{SOLUTION}
Probability mass function of a trinomial distribution is
\begin{multline}
   \p{x}{y} \\=\brak{\frac{7!}{x!y!(7-x-y)!}}(0.6)^x(0.2)^y(0.2)^{7-x-y}\\
  \nonumber  =\brak{\frac{7!}{x!(7-x)!}\frac{(7-x)!}{y!(7-x-y)!}}(0.6)^x(0.2)^y(0.2)^{7-x-y}
\end{multline}
\begin{equation}
    \p{x}{y}=\comb{7}{x}\comb{7-x}{y}(0.6)^x(0.2)^y(0.2)^{7-x-y}\label{1}
\end{equation}
Using \eqref{1}, $\q{x}$ is 
\begin{align}
   \nonumber \q{x}&=\sum_{y=0}^{7-x} \p{x}{y}\\
  \nonumber &=\comb{7}{x}(0.6)^x\,\sum_{y=0}^{7-x}\, \comb{7-x}{y} (0.2)^y(0.2)^{7-x-y} \\
  \nonumber  &=\comb{7}{x}(0.6)^x\,{(0.4)^{7-x}}\\
    \q{x}&=\comb{7}{x}(0.6)^x\,{(0.4)^{7-x}}\label{2}
\end{align}
We have to find $\mean{Y|X=3}$ ,
\begin{align}
    \mean{Y|X=3}&=\sum_{y=0}^4 \: y P(Y=y|X=3)\\
    \mean{Y|X=3}&=\sum_{y=0}^4 y \brak{  \frac{ \p{3}{y}}{P(X=3)} }\label{3}
\end{align}
By taking X=3 in \eqref{1} and \eqref{2}  to use in \eqref{3},
\begin{align}
   \nonumber \mean{Y|X=3}&=\sum_{y=0}^4 y \brak{ \frac{ \p{3}{y}}{P(X=3)}}\\
  \nonumber  &=\sum_{y=0}^4 y   \brak{\frac{\comb{7}{3}\comb{4}{y}(0.6)^3(0.2)^y(0.2)^{4-y}}{\comb{7}{3}(0.6)^3\,{(0.4)^{4}}}}\\
 \nonumber &=\sum_{y=0}^4 y   \brak{\frac{\comb{4}{y}(0.2)^{4}}{{(0.4)^{4}}}}\\
 &=\sum_{y=0}^4 {\frac{y(\comb{4}{y})}{{16}}}\\
  \mean{Y|X=3}&=\frac{1}{16}\sum_{y=0}^4\, {y(\comb{4}{y})}=2
\end{align}
Therefore the value of $\mean{Y|X=3}=2$.
\end{document}
