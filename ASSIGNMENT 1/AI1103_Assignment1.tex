\documentclass[journal,12pt,twocolumn]{IEEEtran}

\usepackage{setspace}
\usepackage{gensymb}
\singlespacing
\usepackage[cmex10]{amsmath}

\usepackage{amsthm}

\usepackage{mathrsfs}
\usepackage{txfonts}
\usepackage{stfloats}
\usepackage{bm}
\usepackage{cite}
\usepackage{cases}
\usepackage{subfig}

\usepackage{longtable}
\usepackage{multirow}

\usepackage{enumitem}
\usepackage{mathtools}
\usepackage{steinmetz}
\usepackage{tikz}
\usepackage{circuitikz}
\usepackage{verbatim}
\usepackage{tfrupee}
\usepackage[breaklinks=true]{hyperref}
\usepackage{graphicx}
\usepackage{tkz-euclide}

\usetikzlibrary{calc,math}
\usepackage{listings}
    \usepackage{color}                                            %%
    \usepackage{array}                                            %%
    \usepackage{longtable}                                        %%
    \usepackage{calc}                                             %%
    \usepackage{multirow}                                         %%
    \usepackage{hhline}                                           %%
    \usepackage{ifthen}                                           %%
    \usepackage{lscape}     
\usepackage{multicol}
\usepackage{chngcntr}

\DeclareMathOperator*{\Res}{Res}

\renewcommand\thesection{\arabic{section}}
\renewcommand\thesubsection{\thesection.\arabic{subsection}}
\renewcommand\thesubsubsection{\thesubsection.\arabic{subsubsection}}

\renewcommand\thesectiondis{\arabic{section}}
\renewcommand\thesubsectiondis{\thesectiondis.\arabic{subsection}}
\renewcommand\thesubsubsectiondis{\thesubsectiondis.\arabic{sub subsection}}


\hyphenation{optical networks semiconduc-tor}
\def\inputGnumericTable{}                                 %%

\lstset{
%language=C,
frame=single, 
breaklines=true,
columns=fullflexible
}
\date{March 2021}

\begin{document}

\newcommand{\BEQA}{\begin{eqnarray}}
\newcommand{\EEQA}{\end{eqnarray}}
\newcommand{\define}{\stackrel{\triangle}{=}}
\bibliographystyle{IEEEtran}
\raggedbottom
\setlength{\parindent}{0pt}
\providecommand{\mbf}{\mathbf}
\providecommand{\pr}[1]{\ensuremath{\Pr\left(#1\right)}}
\providecommand{\qfunc}[1]{\ensuremath{Q\left(#1\right)}}
\providecommand{\sbrak}[1]{\ensuremath{{}\left[#1\right]}}
\providecommand{\lsbrak}[1]{\ensuremath{{}\left[#1\right.}}
\providecommand{\rsbrak}[1]{\ensuremath{{}\left.#1\right]}}
\providecommand{\brak}[1]{\ensuremath{\left(#1\right)}}
\providecommand{\lbrak}[1]{\ensuremath{\left(#1\right.}}
\providecommand{\rbrak}[1]{\ensuremath{\left.#1\right)}}
\providecommand{\cbrak}[1]{\ensuremath{\left\{#1\right\}}}
\providecommand{\lcbrak}[1]{\ensuremath{\left\{#1\right.}}
\providecommand{\rcbrak}[1]{\ensuremath{\left.#1\right\}}}
\theoremstyle{remark}
\newtheorem{rem}{Remark}
\newcommand{\sgn}{\mathop{\mathrm{sgn}}}
\providecommand{\abs}[1]{\vert#1\vert}
\providecommand{\res}[1]{\Res\displaylimits_{#1}} 
\providecommand{\norm}[1]{\lVert#1\rVert}
%\providecommand{\norm}[1]{\lVert#1\rVert}
\providecommand{\mtx}[1]{\mathbf{#1}}
\providecommand{\mean}[1]{E[ #1 ]}
\providecommand{\fourier}{\overset{\mathcal{F}}{ \rightleftharpoons}}
%\providecommand{\hilbert}{\overset{\mathcal{H}}{ \rightleftharpoons}}
\providecommand{\system}{\overset{\mathcal{H}}{ \longleftrightarrow}}
	%\newcommand{\solution}[2]{\textbf{Solution:}{#1}}
\newcommand{\solution}{\noindent \textbf{Solution: }}
\newcommand{\cosec}{\,\text{cosec}\,}
\providecommand{\dec}[2]{\ensuremath{\overset{#1}{\underset{#2}{\gtrless}}}}
\newcommand{\myvec}[1]{\ensuremath{\begin{pmatrix}#1\end{pmatrix}}}
\newcommand{\mydet}[1]{\ensuremath{\begin{vmatrix}#1\end{vmatrix}}}
\numberwithin{equation}{subsection}
\makeatletter
\@addtoreset{figure}{problem}
\makeatother
\let\StandardTheFigure\thefigure
\let\vec\mathbf
\renewcommand{\thefigure}{\theproblem}
\def\putbox#1#2#3{\makebox[0in][l]{\makebox[#1][l]{}\raisebox{\baselineskip}[0in][0in]{\raisebox{#2}[0in][0in]{#3}}}}
     \def\rightbox#1{\makebox[0in][r]{#1}}
     \def\centbox#1{\makebox[0in]{#1}}
     \def\topbox#1{\raisebox{-\baselineskip}[0in][0in]{#1}}
     \def\midbox#1{\raisebox{-0.5\baselineskip}[0in][0in]{#1}}
\vspace{3cm}
\title{Assignment 1}
\author{Adhvik Mani Sai Murarisetty - AI20BTECH11015}
\maketitle
\newpage
\bigskip
\renewcommand{\thefigure}{\theenumi}
\renewcommand{\thetable}{\theenumi}
Download all python codes from 
\begin{lstlisting}
https://github.com/adhvik24/AI1103-PROBABILITY-AND-RANDOM-VARIABLES/blob/main/ASSIGNMENT%201/codes/assign1.py
\end{lstlisting}
%
and latex-tikz codes from 
%
\begin{lstlisting}
https://github.com/adhvik24/AI1103-PROBABILITY-AND-RANDOM-VARIABLES/blob/main/ASSIGNMENT%201/AI1103_Assignment1.tex
\end{lstlisting}
\section{Problem 6.15}
Given two independent events A and B such
that \pr{A} = 0.3, \pr{B} = 0.6. Find
\begin{enumerate}[label={\roman*)}]
    \item \pr{A\,and\,B}
    \item \pr{A\, and\, not\, B}
    \item \pr{A \,or\, B}
    \item \pr{neither\, A \,nor\, B}
\end{enumerate}
\section{Solution}
\begin{enumerate}[label={\roman*)}]
\item
Since the events A and B are independent events, by definition
\begin{align}
    \pr{A\, and\, B} = \pr{AB}= \pr{A}\pr{B}\label{a}
\end{align}
On substituting the values of \pr{A},\pr{B} in \eqref{a}, we get
\begin{align}
    \pr{A\, and\, B} &= \pr{A}\pr{B}\\
    &= (0.3)(0.6)\\
    \implies \pr{A\, and\, B}&=0.18
\end{align}
\item
As the events A and B are independent, then A and B' are also independent.
\begin{align}   
    \implies \pr{A\,and\,not\,B} &=\pr{AB'}\\
&= \pr{A}\pr{B'}\\
\therefore \pr{A\,and\,not\,B} &= \pr{A}\pr{B'}\label{b}
\end{align}
And we know that,
\begin{align}
    \pr{B'}=1-\pr{B}\label{c}
\end{align}
Using \eqref{c} in \eqref{b} we will get,
\begin{align}
   \pr{A\,and\,not\,B} &=\pr{AB'}\\
&= \pr{A}\pr{B'}\\
    \pr{A\,and\,not\,B} &= \pr{A}(1-\pr{B})\label{d}
\end{align}
On substituting the values of \pr{A},\pr{B} in \eqref{d}, we get
\begin{align}
    \pr{A\,and\,not\,B} &= 0.3(1-0.6)\\
    &= (0.3)(0.4)\\
    \implies \pr{A\,and\,not\,B}&= 0.12
\end{align}
\item
\begin{align}
    \pr{A\,or\,B} =\pr{A\,+\,B}\label{e}
\end{align}
We know that,
\begin{align}
    \pr{A\,+\,B} = \pr{A} + \pr{B} -\pr{AB}\label{f}
\end{align}
As events A and B are independent events,
\begin{align}
    \pr{AB}=\pr{A}\pr{B}\label{g}
\end{align}
Using \eqref{g} and \eqref{f} in \eqref{e}, We get
\begin{align}
    \pr{A\,+\,B} = \pr{A} + \pr{B} -\pr{A}\pr{B}\label{h}
\end{align}
On substituting the values of \pr{A},\pr{B} in \eqref{h}, we get
\begin{align}
    \pr{A\,or\,B} &= 0.3 + 0.6 -(0.3)(0.6)\\
    &= 0.9-0.18\\
    \implies \pr{A\,or\,B} &= 0.72\label{i}
\end{align}
\item
\begin{align}
    \pr{neither\, A\, nor\, B} &= \pr{A'B'}\\
    &=\pr{(A\,+\,B)'}\\
    \pr{neither\, A\, nor\, B}&=1-\pr{A\,+\,B}\label{j}
\end{align}
From \eqref{i},
\begin{align}
    \pr{A\,or\,B} =\pr{A\,+\,B}=0.72\label{k}
\end{align}
Using \eqref{k} in \eqref{j}, We get
\begin{align}
    \pr{neither\, A\, nor\, B}&=1-0.72\\
   \implies \pr{neither\, A\, nor\, B}&=0.28
\end{align}
\end{enumerate}
\end{document}
