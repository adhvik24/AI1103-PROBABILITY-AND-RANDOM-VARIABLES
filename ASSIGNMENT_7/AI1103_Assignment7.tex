\documentclass[journal,12pt,twocolumn]{IEEEtran}

\usepackage{setspace}
\usepackage{gensymb}
\singlespacing
\usepackage[cmex10]{amsmath}

\usepackage{amsthm}

\usepackage{mathrsfs}
\usepackage{txfonts}
\usepackage{stfloats}
\usepackage{bm}
\usepackage{cite}
\usepackage{cases}
\usepackage{subfig}

\usepackage{longtable}
\usepackage{multirow}
\usepackage{enumitem}
\usepackage{mathtools}
\usepackage{steinmetz}
\usepackage{tikz}
\usepackage{circuitikz}
\usepackage{verbatim}
\usepackage{tfrupee}
\usepackage[breaklinks=true]{hyperref}
\usepackage{graphicx}
\usepackage{tkz-euclide}

\usetikzlibrary{calc,math}
\usepackage{listings}
    \usepackage{color}                                            %%
    \usepackage{array}                                            %%
    \usepackage{longtable}                                        %%
    \usepackage{calc}                                             %%
    \usepackage{multirow}                                         %%
    \usepackage{hhline}                                           %%
    \usepackage{ifthen}                                           %%
    \usepackage{lscape}     
\usepackage{multicol}
\usepackage{chngcntr}

\DeclareMathOperator*{\Res}{Res}

\renewcommand\thesection{\arabic{section}}
\renewcommand\thesubsection{\thesection.\arabic{subsection}}
\renewcommand\thesubsubsection{\thesubsection.\arabic{subsubsection}}

\renewcommand\thesectiondis{\arabic{section}}
\renewcommand\thesubsectiondis{\thesectiondis.\arabic{subsection}}
\renewcommand\thesubsubsectiondis{\thesubsectiondis.\arabic{sub subsection}}


\hyphenation{optical networks semiconduc-tor}
\def\inputGnumericTable{}                                 %%

\lstset{
%language=C,
frame=single, 
breaklines=true,
columns=fullflexible
}
\date{March 2021}

\begin{document}

\newcommand{\BEQA}{\begin{eqnarray}}
\newcommand{\EEQA}{\end{eqnarray}}
\newcommand{\define}{\stackrel{\triangle}{=}}
\bibliographystyle{IEEEtran}
\raggedbottom
\setlength{\parindent}{0pt}
\providecommand{\mbf}{\mathbf}
\providecommand{\pr}[1]{\ensuremath{\Pr\left(#1\right)}}
\providecommand{\qfunc}[1]{\ensuremath{Q\left(#1\right)}}
\providecommand{\fn}[1]{\ensuremath{f\left({#1}\right)}}
\providecommand{\e}[1]{\ensuremath{E\left(#1\right)}}
\providecommand{\sbrak}[1]{\ensuremath{{}\left[#1\right]}}
\providecommand{\lsbrak}[1]{\ensuremath{{}\left[#1\right.}}
\providecommand{\rsbrak}[1]{\ensuremath{{}\left.#1\right]}}
\providecommand{\brak}[1]{\ensuremath{\left(#1\right)}}
\providecommand{\lbrak}[1]{\ensuremath{\left(#1\right.}}
\providecommand{\rbrak}[1]{\ensuremath{\left.#1\right)}}
\providecommand{\cbrak}[1]{\ensuremath{\left\{#1\right\}}}
\providecommand{\lcbrak}[1]{\ensuremath{\left\{#1\right.}}
\providecommand{\rcbrak}[1]{\ensuremath{\left.#1\right\}}}
\theoremstyle{remark}
\newtheorem{rem}{Remark}
\newcommand{\sgn}{\mathop{\mathrm{sgn}}}
\newcommand{\comb}[2]{{}^{#1}\mathrm{C}_{#2}}
\providecommand{\abs}[1]{\vert#1\vert}
\providecommand{\res}[1]{\Res\displaylimits_{#1}} 
\providecommand{\norm}[1]{\lVert#1\rVert}
%\providecommand{\norm}[1]{\lVert#1\rVert}
\providecommand{\mtx}[1]{\mathbf{#1}}
\providecommand{\mean}[1]{E\sbrak{ #1 }}
\providecommand{\fourier}{\overset{\mathcal{F}}{ \rightleftharpoons}}
%\providecommand{\hilbert}{\overset{\mathcal{H}}{ \rightleftharpoons}}
\providecommand{\system}{\overset{\mathcal{H}}{ \longleftrightarrow}}
	%\newcommand{\solution}[2]{\textbf{Solution:}{#1}}
\newcommand{\solution}{\noindent \textbf{Solution: }}
\newcommand{\cosec}{\,\text{cosec}\,}
\providecommand{\dec}[2]{\ensuremath{\overset{#1}{\underset{#2}{\gtrless}}}}
\newcommand{\myvec}[1]{\ensuremath{\begin{pmatrix}#1\end{pmatrix}}}
\newcommand{\mydet}[1]{\ensuremath{\begin{vmatrix}#1\end{vmatrix}}}
\numberwithin{equation}{subsection}
\makeatletter
\vspace{3cm}
\title{Assignment 7}
\author{Adhvik Mani Sai Murarisetty - AI20BTECH11015}
\maketitle
\newpage
\bigskip
\renewcommand{\thetable}{\theenumi}

%
Download latex-tikz codes from 
%
\begin{lstlisting}
https://github.com/adhvik24/AI1103-PROBABILITY-AND-RANDOM-VARIABLES/blob/main/ASSIGNMENT_7/AI1103_Assignment7.tex
\end{lstlisting}
\section{GATE 2021 (ME-SET1), Q.20 (ME section)}
Robot Ltd. wishes to maintain enough safety
stock during the lead time period between
starting a new production run and its completion
such that the probability of satisfying the
customer demand during the lead time period
is 95\%. The lead time periods is 5 days and
daily customer demand can be assumed to follow
the Gaussian (normal) distribution with mean
50 units and a standard deviation of 10 units.
Using $\phi^{-1}$(0.95) =1.64 , where $\phi$ represents the
cumulative distribution function of the standard
normal random variable, the amount of safety
stock that must be maintained by Robot Ltd. to
achieve this demand fulfillment probability for
the lead time period is \rule{1cm}{0.15mm}  units (round off to
two decimal places).
\section{SOLUTION}
Let X be the normal R.V denoting the required amount of stock of customer demand over the lead time. 

Let $X_1$ be the normal R.V denoting daily customer demand with mean and standard deviation as follows,
\begin{table}[htp]
\centering
\begin{tabular}{ |c|c|c|} 
\hline
parameter & value(in units) \\
\hline
Mean of $X_1\;(\mu)$& 50 \\
\hline
Standard deviation of $X_1\;(\sigma)$& 10\\
\hline
\end{tabular}
\caption{mean and standard deviation of $X_1$}
\label{table1}
\end{table}

Probability of satisfying customer demand is 0.95.

Let Z be a standard normal R.V such that,
\begin{align}
    Z=\frac{X_1-\mu}{\sigma} \label{3}
\end{align}
Referring table\eqref{table1} to use in \eqref{3},
\begin{align}
    Z=\frac{X_1-50}{10}\label{4}
\end{align}
Given that,
\begin{align}
    \phi^{-1}(0.95)&=1.64\\
    \implies \phi(1.64)&=0.95\\
    \phi(1.64)&=\pr{Z\le1.64}=0.95\\
    \implies Z\le1.64&\iff\frac{X_1-50}{10}\le1.64\\
    \implies X_1-50 &\le 1.64(10)\\
    \therefore X_1\le66.4
\end{align}
The demand in one day is independent of demand in the other day and the lead time is 5 days.
\begin{align}
    \implies X=5(X_1)=5(66.4)=332
\end{align}
Therefore the amount of safety
stock that must be maintained by Robot Ltd. to
achieve this demand fulfillment probability for
the lead time period is 332 units.
\end{document}

