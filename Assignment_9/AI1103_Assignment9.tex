\documentclass[journal,12pt,twocolumn]{IEEEtran}

\usepackage{setspace}
\usepackage{gensymb}
\singlespacing
\usepackage[cmex10]{amsmath}

\usepackage{amsthm}

\usepackage{mathrsfs}
\usepackage{txfonts}
\usepackage{stfloats}
\usepackage{bm}
\usepackage{cite}
\usepackage{cases}
\usepackage{subfig}

\usepackage{longtable}
\usepackage{multirow}
\usepackage{enumitem}
\usepackage{mathtools}
\usepackage{steinmetz}
\usepackage{tikz}
\usepackage{circuitikz}
\usepackage{verbatim}
\usepackage{tfrupee}
\usepackage[breaklinks=true]{hyperref}
\usepackage{graphicx}
\usepackage{tkz-euclide}

\usetikzlibrary{calc,math}
\usepackage{listings}
    \usepackage{color}                                            %%
    \usepackage{array}                                            %%
    \usepackage{longtable}                                        %%
    \usepackage{calc}                                             %%
    \usepackage{multirow}                                         %%
    \usepackage{hhline}                                           %%
    \usepackage{ifthen}                                           %%
    \usepackage{lscape}     
\usepackage{multicol}
\usepackage{chngcntr}

\DeclareMathOperator*{\Res}{Res}

\renewcommand\thesection{\arabic{section}}
\renewcommand\thesubsection{\thesection.\arabic{subsection}}
\renewcommand\thesubsubsection{\thesubsection.\arabic{subsubsection}}

\renewcommand\thesectiondis{\arabic{section}}
\renewcommand\thesubsectiondis{\thesectiondis.\arabic{subsection}}
\renewcommand\thesubsubsectiondis{\thesubsectiondis.\arabic{sub subsection}}


\hyphenation{optical networks semiconduc-tor}
\def\inputGnumericTable{}                                 %%

\lstset{
%language=C,
frame=single, 
breaklines=true,
columns=fullflexible
}
\date{March 2021}

\begin{document}

\newcommand{\BEQA}{\begin{eqnarray}}
\newcommand{\EEQA}{\end{eqnarray}}
\newcommand{\define}{\stackrel{\triangle}{=}}
\bibliographystyle{IEEEtran}
\raggedbottom
\setlength{\parindent}{0pt}
\providecommand{\mbf}{\mathbf}
\providecommand{\pr}[1]{\ensuremath{\Pr\left(#1\right)}}
\providecommand{\qfunc}[1]{\ensuremath{Q\left(#1\right)}}
\providecommand{\fn}[1]{\ensuremath{f\left({#1}\right)}}
\providecommand{\e}[1]{\ensuremath{E\left(#1\right)}}
\providecommand{\sbrak}[1]{\ensuremath{{}\left[#1\right]}}
\providecommand{\lsbrak}[1]{\ensuremath{{}\left[#1\right.}}
\providecommand{\rsbrak}[1]{\ensuremath{{}\left.#1\right]}}
\providecommand{\brak}[1]{\ensuremath{\left(#1\right)}}
\providecommand{\lbrak}[1]{\ensuremath{\left(#1\right.}}
\providecommand{\rbrak}[1]{\ensuremath{\left.#1\right)}}
\providecommand{\cbrak}[1]{\ensuremath{\left\{#1\right\}}}
\providecommand{\lcbrak}[1]{\ensuremath{\left\{#1\right.}}
\providecommand{\rcbrak}[1]{\ensuremath{\left.#1\right\}}}
\theoremstyle{remark}
\newtheorem{rem}{Remark}
\newcommand{\sgn}{\mathop{\mathrm{sgn}}}
\newcommand{\comb}[2]{{}^{#1}\mathrm{C}_{#2}}
\providecommand{\abs}[1]{\vert#1\vert}
\providecommand{\res}[1]{\Res\displaylimits_{#1}} 
\providecommand{\norm}[1]{\lVert#1\rVert}
%\providecommand{\norm}[1]{\lVert#1\rVert}
\providecommand{\mtx}[1]{\mathbf{#1}}
\providecommand{\mean}[1]{E\sbrak{ #1 }}
\providecommand{\fourier}{\overset{\mathcal{F}}{ \rightleftharpoons}}
%\providecommand{\hilbert}{\overset{\mathcal{H}}{ \rightleftharpoons}}
\providecommand{\system}{\overset{\mathcal{H}}{ \longleftrightarrow}}
	%\newcommand{\solution}[2]{\textbf{Solution:}{#1}}
\newcommand{\solution}{\noindent \textbf{Solution: }}
\newcommand{\cosec}{\,\text{cosec}\,}
\providecommand{\dec}[2]{\ensuremath{\overset{#1}{\underset{#2}{\gtrless}}}}
\newcommand{\myvec}[1]{\ensuremath{\begin{pmatrix}#1\end{pmatrix}}}
\newcommand{\mydet}[1]{\ensuremath{\begin{vmatrix}#1\end{vmatrix}}}
\numberwithin{equation}{subsection}
\makeatletter
\vspace{3cm}
\title{Assignment 9}
\author{Adhvik Mani Sai Murarisetty - AI20BTECH11015}
\maketitle
\newpage
\bigskip
\renewcommand{\thetable}{\theenumi}

%
Download latex-tikz codes from 
%
\begin{lstlisting}
https://github.com/adhvik24/AI1103-PROBABILITY-AND-RANDOM-VARIABLES/blob/main/Assignment_9/AI1103_Assignment9.tex
\end{lstlisting}
\section{GATE 2020 (ST), Q.23 (statistics section)}
Let $X_1,X_2,\cdots,X_n$ be a random sample of size n $(n\ge2)$ from an exponential distribution with the probability density function
\begin{align}
\fn{x,\theta}=
\begin{cases}
e^{-(x-2\theta)}, & x>2\theta
\\
0, & otherwise
\end{cases}
\end{align}
where $\theta\in\brak{0,\infty}$. If $X_{(1)}=min\cbrak{X_1,X_2,\cdots,X_n}$ then the conditional expectation 
\begin{align}
     \nonumber   \mean{\frac{1}{\theta}\brak{X_{(1)}-\frac{1}{n}}|X_1-X_2=2}=\rule{1.5cm}{0.15mm}
\end{align}
\section{SOLUTION}
\textbf{DEFINITIONS:}
\begin{enumerate}
    \item \textbf{Completeness: }The statistic T is said to be complete for the distribution of X if, for every measurable function g
    if 
    \begin{align}
        E(g(T))=0\implies P(g(T)=0)=1\;\forall\theta
    \end{align}
    \item\textbf{Sufficiency: }
    Let $f(x,\theta)$ be the joint pdf of the sample X. A statistic T is sufficient for $\theta$ iff there are functions h (does not depend on $\theta$)and g ( depends on $\theta$) on the range of T such that
\begin{align}
    f(x,\theta)=g(T(x),\theta)\;h(x)
\end{align}
    \item\textbf{Basu's Theorem: }
    If T(X) is complete and sufficient, and S(X) is ancillary, then S(X) and T(X) are independent for all $\theta$.
    
    $\implies$complete sufficient statistic is independent of any ancillary statistic.
\end{enumerate}

Given PDF of the distribution as,
\begin{align}
\fn{x,\theta}=
\begin{cases}
e^{-(x-2\theta)}, & x>2\theta
\\
0, & otherwise
\end{cases}\label{2}
\end{align}
Then CDF of the distribution given is,
\begin{align}
    F(x,\theta)&=\int_{-\infty}^x \fn{x,\theta} dx\label{3}
\end{align}
Using \eqref{2} in \eqref{3},
\begin{align}
F(x,\theta)&=
\begin{cases}
0, & x<2\theta
\\
1-e^{-(x-2\theta)}, & x>2\theta
\end{cases}\label{4}
\end{align}
As given $X_{(1)}=min\cbrak{X_1,X_2,\cdots,X_n}$,\\
Let us find CDF of $X_{(1)}$,
\begin{align}
  \nonumber  F_{X_{(1)}}(x,\theta)&=\pr{X_{(1)}\le x}\\
  \nonumber  &=\pr{at\,least\,one\, of\, X_1,X_2,\cdots,X_n \le x}\\
 \nonumber   &=1-\pr{X_{(1)}>x}\\
  \nonumber  &=1-\pr{X_1>x,X_2>x,\cdots,X_n>1}\\
  \nonumber &=1-\pr{X_1>x}\cdots \pr{X_n>x}\\
    &=1-\brak{1-F(x,\theta)}^n\label{5}
\end{align}
Using \eqref{4} in \eqref{5},
\begin{align}
F_{X_{(1)}}(x,\theta)&=
\begin{cases}
0, & x<2\theta
\\
1-e^{-n(x-2\theta)}, & x>2\theta
\end{cases}\label{6}
\end{align}
Using CDF of $X_{(1)}$ to find PDF of ${X_{(1)}}$,
\begin{align}
    f_{X_{(1)}}(x,\theta)= \dfrac{d}{dx}\;F_{X_{(1)}}(x,\theta)\label{7}
\end{align}
Using \eqref{6} in \eqref{7}, PDF of $X_{(1)}$ is
\begin{align}
f_{X_{(1)}}{(x,\theta)}=
\begin{cases}
n e^{-n(x-2\theta)}, & x>2\theta
\\
0, & otherwise
\end{cases}\label{8}
\end{align}
$X_{(1)},\cdots,X_{(n)}$ are ordered statistics of $X_1,\cdots,X_n$.

Where $X_{(k)}$ is kth order statistic of $X_1,\cdots,X_n$.
\begin{align}
    \implies \sum\limits_{i=1}^n X_i = \sum\limits_{i=1}^n X_{(i)}\label{a_2}
\end{align}
Some results that we use in future:
\begin{enumerate}
    \item Ordered statistics are complete and sufficient statistic of X.
    
    \textbf{Proof:} 

    Let $E[g(X_{(1)})]=0$,
    \begin{align}
        \implies\int\limits_{-\infty}^{\infty} g(x) f_{X_{(1)}}(x) dx&=0\\
        \int\limits_{2\theta}^{\infty} g(x) n e^{-n(x-2\theta)} dx &=0\\
        \int\limits_{2\theta}^{\infty} g(x) e^{-n(x-2\theta)} dx &=0\label{a_1}
    \end{align}
    differentiating w.r.t $\theta$ on both sides in \eqref{a_1},
    \begin{align}
      \nonumber  \dfrac{d}{dx} \int\limits_{2\theta}^{\infty} g(x) e^{-n(x-2\theta)} dx =0\\
     \nonumber   \dfrac{d}{dx} \brak{\int\limits_{2\theta}^{\infty} g(x) e^{-nx} dx} e^{2n\theta} =0\\
     \nonumber   2n e^{2n\theta}\int\limits_{2\theta}^{\infty} g(x) e^{-nx} dx+e^{2n\theta}(2)g(2\theta)e^{-2n\theta}=0\\
     \nonumber2n(0)+2g(2\theta)=0\implies g(2\theta)=0
    \end{align}
    $\implies\;X_{(1)}$ is complete statistics.
    
    Using \eqref{a_2} in \eqref{a_3}
    \begin{align}
        f_X(x,\theta)&=f(x_1,\theta)f(x_2,\theta)\cdots f(x_n,\theta)\\
        &=e^{-(x_1-2\theta)}e^{-(x_2-2\theta)}\cdots e^{-(x_n-2\theta)}\\
        &=e^{-\brak{\sum\limits_{i=1}^n x_i-2n\theta}}=e^{-\brak{\sum\limits_{i=1}^n x_{(i)}-2n\theta}}\label{a_3}\\
        &=\underbrace{ \prod\limits_{j=1}^n e^{-(x_{(j)}-2\theta)}}_\text{g}\times\underbrace{\brak{1}}_\text{h}
    \end{align}
    $\therefore$ Ordered statistics of X are sufficient statistics for $\theta$.
    
    $\therefore\;X_{(1)}$ is complete and sufficient statistics of $\theta$. 
    \item $X_1-X_2$ is ancillary of $\theta$.
    
    \textbf{Proof: }Let U=$X_1-X_2$ then,
    \begin{align}
     \nonumber   F_U(x)&=\pr{X_1-X_2<x}\\
    \nonumber
    &=\int\limits_{-\infty}^{\infty}\pr{X_1<x+k}\pr{X_2>k} dk\\
  \nonumber      &= \int\limits_{2\theta}^{\infty}\brak{1-e^{-\brak{x+k-2\theta}}}\brak{e^{-\brak{k-2\theta}}} dk\\
 \nonumber &=\int\limits_{2\theta}^{\infty}e^{-\brak{k-2\theta}}-e^{-\brak{2k+x-2\theta}} dk\\
\nonumber &= \sbrak{\frac{e^{-\brak{k-2\theta}}}{-1}-\frac{e^{-\brak{2k+x-2\theta}}}{-2}}_{2\theta}^{\infty}\\
  \nonumber  &=(0-0)-\brak{-1+\frac{e^{-x}}{2}}\\
    F_U(x)&= 1-\frac{e^{-x}}{2}\\
    \implies f_U(x)&=\dfrac{d}{dx}F_U(x)\\
    &=\frac{e^{-x}}{2}
    \end{align}
    $\therefore\;U=X_1-X_2$ is an ancillary statistic of $\theta$.
\end{enumerate}

Let U be a random variable such that $U=X_1-X_2$.
\begin{multline}
    \mean{\frac{1}{\theta}\brak{X_{(1)}-\frac{1}{n}}|X_1-X_2=2}\\=\mean{\frac{1}{\theta}\brak{X_{(1)}-\frac{1}{n}}|U=2}\label{1}
\end{multline}
As $X_1,X_2,\cdots,X_n$ are independent and from Basu's theorem $X_{(1)}$ and U are also independent. \\As we know that if X and Y are independent then $\mean{X|Y}=\mean{X}$. 
Using this in \eqref{1}
\begin{align}
    \mean{\frac{1}{\theta}\brak{X_{(1)}-\frac{1}{n}}|U=2}
    &=\mean{\frac{1}{\theta}\brak{X_{(1)}-\frac{1}{n}}}\\
    &=\frac{1}{\theta}\brak{\mean{X_{(1)}}-\frac{1}{n}}\label{a}
\end{align}
We have to find expectation of $X_{(1)}$,
\begin{align}
   \mean{X_{(1)}}&=\int_{-\infty}^{\infty} x f_{X_{(1)}}{(x,\theta)} dx \label{9}
\end{align}
Using \eqref{8} in \eqref{9}.
\begin{align}
\nonumber \mean{X_{(1)}}&=\int_{2\theta}^{\infty} n x\,e^{-(x-2\theta)n} dx\\
  &= e^{2n\theta}\int_{2\theta}^{\infty} n x\,e^{-n x} dx \label{10}
\end{align}
Using integration by parts in \eqref{10},
\begin{align}
    \nonumber \mean{X_{(1)}}
  &= e^{2n\theta}\int_{2\theta}^{\infty}nx\,e^{-n x}dx\\
      \nonumber &= e^{2n\theta}\brak{\sbrak{n x \frac{e^{-n x}}{-n}}_{2\theta}^{\infty}-\int_{2\theta}^{\infty} n \frac{e^{-n x}}{-n}dx}\\
  \nonumber &= e^{2n\theta}\brak{\sbrak{n x \frac{e^{-n x}}{-n}}_{2\theta}^{\infty}+\sbrak{\frac{e^{-n x}}{-n}}_{2\theta}^{\infty}}\\
  \nonumber &= e^{2n\theta}\brak{2\theta e^{-2n\theta}+\frac{e^{-2n\theta}}{n}}\\
  \mean{X_{(1)}} &= 2\theta+\frac{1}{n}\label{11}
\end{align}
Use \eqref{11} in \eqref{a},
\begin{align}
   \nonumber \mean{\frac{1}{\theta}\brak{X_{(1)}-\frac{1}{n}}|U=2}
    &=\frac{1}{\theta}\brak{\mean{X_{(1)}}-\frac{1}{n}}\\
   \nonumber &=\frac{1}{\theta}\brak{2\theta+\frac{1}{n}-\frac{1}{n}}\\
   \mean{\frac{1}{\theta}\brak{X_{(1)}-\frac{1}{n}}|U=2} &= 2\label{b}
\end{align}
Using \eqref{b} in \eqref{1},
\begin{align}
     \nonumber\therefore   \mean{\frac{1}{\theta}\brak{X_{(1)}-\frac{1}{n}}|X_1-X_2=2}=2
\end{align}
\end{document}
